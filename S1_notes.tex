%! Author = julesdupont
%! Date = 24/09/2024

% Preamble
\documentclass[11pt]{article}

% Packages
\usepackage{amsmath}
\usepackage{listings}
\usepackage{xcolor}
\usepackage{hyperref}

\lstdefinestyle{mystyle}{
    basicstyle=\ttfamily\footnotesize,
    breaklines=true,
    numberstyle=\tiny\color{gray},
    numbers=left,
    numbersep=5pt
}

\lstset{style=mystyle}


% Document
\begin{document}

    \title{\small INTRODUCTION TO C++ \\ \LARGE LESSON 1: Introduction}
    \author{Cédric \textsc{Zanni}}

    \maketitle

    \section{Introduction}

    C++ is widely used, from embedded systems to Python reference implementation.

    \section{Modern C++}

    \begin{itemize}
        \item static type safety
        \item resources safety
        \item abstraction
        \item encapsulation; invariants
        \item generic programming
        \item multiparadigm
    \end{itemize}

    \section{C(++): a compiled language}

    With both C and C++, source code needs to be compiled into assembly language, which will then be translated to machine code (binary).

    \section{Syntax - Delaration and instruction}

    \subsection{Declare variables}
    \begin{lstlisting}[language=C]
int capacity;
float height;
    \end{lstlisting}

    \subsection{Declare and initialize variables (adivsed!)}
    \begin{lstlisting}[language=C]
int capacity = 1;
float height = 0.5F;
    \end{lstlisting}

    \subsection{Instructions}
    \begin{lstlisting}[language=C]
...
    \end{lstlisting}


\end{document}